\documentclass[11pt]{article}
\usepackage[paperheight=1.3in,paperwidth=7.4in,margin=0.01in]{geometry}

\usepackage{setspace}
\usepackage{amsmath,amssymb}
\usepackage{amsthm}
\usepackage{fancybox}
\usepackage{url}
\usepackage{enumitem}
\usepackage{multirow}
\usepackage{color}
\usepackage{graphicx}
\usepackage{setspace}
\usepackage{comment}
\usepackage{bm}
\usepackage{mathtools}
\mathtoolsset{showonlyrefs=true}

\usepackage{natbib}
\usepackage{xr}
\externaldocument{ordinalT}
%\usepackage{icml2020}

\usepackage{pifont}
\newcommand{\cmark}{\ding{51}}%
\newcommand{\xmark}{\ding{55}}%

%\input macros.tex

\def\@normalsize{\@setsize\normalsize{11pt}\xpt\@xpt}

\begin{document}
\begin{table}
\normalsize
\begin{tabular}{l|ccc}
&Bhaskar [2016]&Ghadermarzy et al.\ [2018] &This paper\\
\hline 
Higher-order tensors ($K\geq 3$) & \xmark&\cmark& \cmark\\
\hline
Multi-level categories ($L\geq 3$)& \cmark& \xmark&\cmark\\
 \hline
 Error rate for tensor denoising&$d^{-1}$ for $K=2$& $d^{-(K-1)/2}$&$d^{-(K-1)}$\\
\hline
Optimal guarantee under exact low-rankness& unknown & \xmark & \cmark\\
\hline
Sample complexity for tensor completion&$d^K$& $Kd$&$Kd$\\
\hline
\end{tabular}
\end{table}

%\bibliography{tensor_wang}
\bibliographystyle{icml2020}


\end{document}